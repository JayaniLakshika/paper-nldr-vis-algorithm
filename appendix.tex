% Options for packages loaded elsewhere
\PassOptionsToPackage{unicode}{hyperref}
\PassOptionsToPackage{hyphens}{url}
\PassOptionsToPackage{dvipsnames,svgnames,x11names}{xcolor}
%
\documentclass[
  12pt]{article}

\usepackage{amsmath,amssymb}
\usepackage{iftex}
\ifPDFTeX
  \usepackage[T1]{fontenc}
  \usepackage[utf8]{inputenc}
  \usepackage{textcomp} % provide euro and other symbols
\else % if luatex or xetex
  \usepackage{unicode-math}
  \defaultfontfeatures{Scale=MatchLowercase}
  \defaultfontfeatures[\rmfamily]{Ligatures=TeX,Scale=1}
\fi
\usepackage{lmodern}
\ifPDFTeX\else  
    % xetex/luatex font selection
\fi
% Use upquote if available, for straight quotes in verbatim environments
\IfFileExists{upquote.sty}{\usepackage{upquote}}{}
\IfFileExists{microtype.sty}{% use microtype if available
  \usepackage[]{microtype}
  \UseMicrotypeSet[protrusion]{basicmath} % disable protrusion for tt fonts
}{}
\makeatletter
\@ifundefined{KOMAClassName}{% if non-KOMA class
  \IfFileExists{parskip.sty}{%
    \usepackage{parskip}
  }{% else
    \setlength{\parindent}{0pt}
    \setlength{\parskip}{6pt plus 2pt minus 1pt}}
}{% if KOMA class
  \KOMAoptions{parskip=half}}
\makeatother
\usepackage{xcolor}
\setlength{\emergencystretch}{3em} % prevent overfull lines
\setcounter{secnumdepth}{5}
% Make \paragraph and \subparagraph free-standing
\makeatletter
\ifx\paragraph\undefined\else
  \let\oldparagraph\paragraph
  \renewcommand{\paragraph}{
    \@ifstar
      \xxxParagraphStar
      \xxxParagraphNoStar
  }
  \newcommand{\xxxParagraphStar}[1]{\oldparagraph*{#1}\mbox{}}
  \newcommand{\xxxParagraphNoStar}[1]{\oldparagraph{#1}\mbox{}}
\fi
\ifx\subparagraph\undefined\else
  \let\oldsubparagraph\subparagraph
  \renewcommand{\subparagraph}{
    \@ifstar
      \xxxSubParagraphStar
      \xxxSubParagraphNoStar
  }
  \newcommand{\xxxSubParagraphStar}[1]{\oldsubparagraph*{#1}\mbox{}}
  \newcommand{\xxxSubParagraphNoStar}[1]{\oldsubparagraph{#1}\mbox{}}
\fi
\makeatother


\providecommand{\tightlist}{%
  \setlength{\itemsep}{0pt}\setlength{\parskip}{0pt}}\usepackage{longtable,booktabs,array}
\usepackage{calc} % for calculating minipage widths
% Correct order of tables after \paragraph or \subparagraph
\usepackage{etoolbox}
\makeatletter
\patchcmd\longtable{\par}{\if@noskipsec\mbox{}\fi\par}{}{}
\makeatother
% Allow footnotes in longtable head/foot
\IfFileExists{footnotehyper.sty}{\usepackage{footnotehyper}}{\usepackage{footnote}}
\makesavenoteenv{longtable}
\usepackage{graphicx}
\makeatletter
\def\maxwidth{\ifdim\Gin@nat@width>\linewidth\linewidth\else\Gin@nat@width\fi}
\def\maxheight{\ifdim\Gin@nat@height>\textheight\textheight\else\Gin@nat@height\fi}
\makeatother
% Scale images if necessary, so that they will not overflow the page
% margins by default, and it is still possible to overwrite the defaults
% using explicit options in \includegraphics[width, height, ...]{}
\setkeys{Gin}{width=\maxwidth,height=\maxheight,keepaspectratio}
% Set default figure placement to htbp
\makeatletter
\def\fps@figure{htbp}
\makeatother

\addtolength{\oddsidemargin}{-.5in}%
\addtolength{\evensidemargin}{-1in}%
\addtolength{\textwidth}{1in}%
\addtolength{\textheight}{1.7in}%
\addtolength{\topmargin}{-1in}%
\usepackage{booktabs}
\usepackage{longtable}
\usepackage{array}
\usepackage{multirow}
\usepackage{wrapfig}
\usepackage{float}
\usepackage{colortbl}
\usepackage{pdflscape}
\usepackage{tabu}
\usepackage{threeparttable}
\usepackage{threeparttablex}
\usepackage[normalem]{ulem}
\usepackage{makecell}
\usepackage{xcolor}
\usepackage{amsmath}
\usepackage{float}
\usepackage{hyperref}
\usepackage[utf8]{inputenc}
\usepackage{bm}
\def\tightlist{}
\usepackage{setspace}
\newcommand\pD{$p\text{-}D$}
\newcommand\kD{$k\text{-}D$}
\newcommand\dD{$d\text{-}D$}
\newcommand\gD{$2\text{-}D$}
\makeatletter
\@ifpackageloaded{caption}{}{\usepackage{caption}}
\AtBeginDocument{%
\ifdefined\contentsname
  \renewcommand*\contentsname{Table of contents}
\else
  \newcommand\contentsname{Table of contents}
\fi
\ifdefined\listfigurename
  \renewcommand*\listfigurename{List of Figures}
\else
  \newcommand\listfigurename{List of Figures}
\fi
\ifdefined\listtablename
  \renewcommand*\listtablename{List of Tables}
\else
  \newcommand\listtablename{List of Tables}
\fi
\ifdefined\figurename
  \renewcommand*\figurename{Figure}
\else
  \newcommand\figurename{Figure}
\fi
\ifdefined\tablename
  \renewcommand*\tablename{Table}
\else
  \newcommand\tablename{Table}
\fi
}
\@ifpackageloaded{float}{}{\usepackage{float}}
\floatstyle{ruled}
\@ifundefined{c@chapter}{\newfloat{codelisting}{h}{lop}}{\newfloat{codelisting}{h}{lop}[chapter]}
\floatname{codelisting}{Listing}
\newcommand*\listoflistings{\listof{codelisting}{List of Listings}}
\makeatother
\makeatletter
\makeatother
\makeatletter
\@ifpackageloaded{caption}{}{\usepackage{caption}}
\@ifpackageloaded{subcaption}{}{\usepackage{subcaption}}
\makeatother

\ifLuaTeX
  \usepackage{selnolig}  % disable illegal ligatures
\fi
\usepackage[]{natbib}
\bibliographystyle{agsm}
\usepackage{bookmark}

\IfFileExists{xurl.sty}{\usepackage{xurl}}{} % add URL line breaks if available
\urlstyle{same} % disable monospaced font for URLs
\hypersetup{
  pdftitle={Appendix: Looking at Non-Linear Dimension Reductions as Models in the Data Space},
  pdfauthor={Jayani P.G. Lakshika; Dianne Cook; Paul Harrison; Michael Lydeamore; Thiyanga S. Talagala},
  colorlinks=true,
  linkcolor={blue},
  filecolor={Maroon},
  citecolor={Blue},
  urlcolor={Blue},
  pdfcreator={LaTeX via pandoc}}



\begin{document}


\def\spacingset#1{\renewcommand{\baselinestretch}%
{#1}\small\normalsize} \spacingset{1}


%%%%%%%%%%%%%%%%%%%%%%%%%%%%%%%%%%%%%%%%%%%%%%%%%%%%%%%%%%%%%%%%%%%%%%%%%%%%%%

\title{\bf Appendix: Looking at Non-Linear Dimension Reductions as
Models in the Data Space}
\author{
Jayani P.G. Lakshika\\
Econometrics \& Business Statistics, Monash University\\
and\\Dianne Cook\\
Econometrics \& Business Statistics, Monash University\\
and\\Paul Harrison\\
MGBP, BDInstitute, Monash University\\
and\\Michael Lydeamore\\
Econometrics \& Business Statistics, Monash University\\
and\\Thiyanga S. Talagala\\
Statistics, University of Sri Jayewardenepura\\
}
\maketitle

\bigskip
\bigskip
\begin{abstract}

\end{abstract}


\newpage
\spacingset{1.9} % DON'T change the spacing!


\begin{table}

\centering{

\centering\begingroup\fontsize{12}{14}\selectfont

\begin{tabular}{>{\raggedright\arraybackslash}p{3cm}>{\raggedright\arraybackslash}p{12cm}}
\toprule
\textbf{Notation} & \textbf{Description}\\
\midrule
$n, p, k$ & number of observations, variables, embedding dimension, respectively\\
$\mathbfit{X}, \mathbfit{x}$ & $p$-dimensional data (population, sample)\\
$\mathbfit{y}$ & $k$-dimensional layout\\
$P$ & orthonormal basis, generating a $d\text{-}dimensional$ linear projection of $p$-dimensional data\\
$T$ & true  model\\
\addlinespace
$g$ & functional mapping from \pD{} to \kD{}, especially as prescribed by NLDR\\
$\mathbfit{\theta}$ & (Hyper-) parameters for NLDR method\\
$r$ & ranges of the embedding components\\
$C^{(j)}$ & $j$-dimensional bin centers\\
$(b_1, b_2)$ & number of bins in each direction\\
\addlinespace
$(a_1, a_2)$ & binwidths, distance between centroids in each direction\\
$(s_1, \ s_2)$ & starting coordinates of the hexagonal grid\\
$q$ & buffer to ensure hexgrid covers data, proportion of data range, 0-1\\
$m$ & number of non-empty bins\\
$b$ & number of  hexagons in the grid\\
\addlinespace
$h$ & hexagonal id\\
$l$ & side length\\
$A$ & area\\
\bottomrule
\end{tabular}
\endgroup{}

}

\caption{\label{tbl-notation}Summary of notation for describing new
methodology.}

\end{table}%

\section{Computing hexagon grid
configurations}\label{computing-hexagon-grid-configurations}

Given range of embedding component, \(r_2\), number of bins along the
x-axis, \(b_1\), and buffer proportion, \(q\), hexagonal starting point
coordinates, \(s_1 = -q\), and \(s_2 = -q \times r_2\). The purpose is
to find width of the hexagon. \(a_1\), and number of bins along the
y-axis, \(b_2\).

\begin{figure}[H]

\centering{

\includegraphics[width=1\textwidth,height=0.3\textheight]{appendix_files/figure-pdf/fig-hex-param-1.pdf}

}

\caption{\label{fig-hex-param}The components of the hexagon grid
illustrating notation.}

\end{figure}%

Geometric arguments give rise to the following constraints.

\(\text{min }a_1 \text{ s.t.}\)

\begin{equation}\phantomsection\label{eq-equation1}{
s_1 - \frac{a_1}{2} < 0,
}\end{equation}

\begin{equation}\phantomsection\label{eq-equation2}{
s_1 + (b_1 - 1) \times a_1 > 1,
}\end{equation}

\begin{equation}\phantomsection\label{eq-equation4}{
s_2 - \frac{a_2}{2} < 0,
}\end{equation}

\begin{equation}\phantomsection\label{eq-equation5}{
s_2 + (b_2 - 1) \times a_2 > r_2.
}\end{equation}

Since \(a_1\) and \(a_2\) are distances,

\[
a_1, a_2 > 0.
\] Also, \((s_1, s_2) \in (-0.1, -0.05)\) as these are multiplicative
offsets in the negative direction.

Equation~\ref{eq-equation1} can be rearranged as,

\[
a_1 > 2s_1
\]

which given \(s_1 < 0\) and \(a_1 > 0\) will \emph{always} be true. The
same logic follows for Equation~\ref{eq-equation4} and substituting
\(a_2 = \frac{\sqrt{3}}{2}a_1\), and \(s_2 = -q \times r_2\) to
Equation~\ref{eq-equation4} can be written as,

\[
a_1 > -\frac{4}{\sqrt{3}}qr_2
\]

Also, substituting \(a_2 = \frac{\sqrt{3}}{2}a_1\),
\(s_2 = -q \times r_2\) and rearranging Equation~\ref{eq-equation5}
gives:

\begin{equation}\phantomsection\label{eq-equation6}{
a_1 > \frac{2(r_2 + qr_2)}{\sqrt{3}(b_2 - 1)}.
}\end{equation}

Similarly, substituting \(s_1 = -q\) Equation~\ref{eq-equation2}
becomes,

\begin{equation}\phantomsection\label{eq-equation7}{
a_1 > \frac{(1 + q)}{(b_1 - 1)}.
}\end{equation}

This is a linear optimization problem. Therefore, the optimal solution
must occur on a vertex. So, by setting Equation~\ref{eq-equation6}
equals to Equation~\ref{eq-equation7} gives,

\[
\frac{2(r_2 + qr_2)}{\sqrt{3}(b_2 - 1)} = \frac{(1 + q)}{(b_1 - 1)}.
\] After rearranging this,

\[
b_2 = 1 + \frac{2r_2(b_1 - 1)}{\sqrt{3}}
\]

and since \(b_2\) should be an integer,

\begin{equation}\phantomsection\label{eq-equation8}{
b_2 = \Big\lceil1 +\frac{2r_2(b_1 - 1)}{\sqrt{3}}\Big\rceil.
}\end{equation}

Furthermore, with known \(b_1\) and \(b_2\), by considering
Equation~\ref{eq-equation2} or Equation~\ref{eq-equation5} as the
\emph{binding} or \emph{active constraint}, can compute \(a_1\).

If Equation~\ref{eq-equation2} is active, then,

\[
\frac{(1 + q)}{(b_1 - 1)} < \frac{2(r_2 + qr_2)}{\sqrt{3}(b_2 - 1)}.
\]

Rearranging this gives,

\[
r_2 > \frac{\sqrt{3}(b_2 - 1)}{2(b_1 - 1)}.
\]

Therefore, if this equality is true, then
\(a_1 = \frac{(1+q)}{(b_1 - 1)}\), otherwise,
\(a_1 = \frac{2r_2(1+q)}{\sqrt{3}(b_2 - 1)}\).

\newpage

\section{Binning the data}\label{binning-the-data}

Points are assigned to the bin they fall into based on the nearest
centroid. If a point is equidistant from multiple centroids, it is
assigned to the centroid with the lowest hexagonal bin ID.

\begin{figure}

\centering{

\includegraphics[width=1\textwidth,height=\textheight]{appendix_files/figure-pdf/fig-assign-data-1.pdf}

}

\caption{\label{fig-assign-data}Binning the data. Points are assigned to
the nearest centroid. If a point is equidistant from multiple centroids,
assigned to the lowest centroid.}

\end{figure}%

\section{Area of a hexagon}\label{area-of-a-hexagon}

The area of a hexagon is defined as \(A = \frac{3\sqrt{3}}{2}l^2\),
where \(l\) is the side length of the hexagon. \(l\) can be computed
using \(a_1\) and \(a_2\).

\begin{center}
\includegraphics[width=1\textwidth,height=0.3\textheight]{appendix_files/figure-pdf/unnamed-chunk-11-1.pdf}
\end{center}

By pythagorean theorem,

\[
l^2 = (\frac{a_1}{2})^2 + (\frac{a_2 - l}{2})^2
\] \[
l^2 - (\frac{a_2 - l}{2})^2 = (\frac{a_1}{2})^2
\] \[
[l - (\frac{a_2 - l}{2})][l + (\frac{a_2 - l}{2})] = (\frac{a_1}{2})^2
\]

\[
3l^2 + 2la_2 - (a_1^2 + a_2^2) = 0
\]

\[
l = \frac{-2a_2 \pm \sqrt{4a_2^2 - 24[-(a_1^2 + a_2^2)]}}{6}
\] \[
l = \frac{-a_2 \pm \sqrt{a_2^2 - 6[-(a_1^2 + a_2^2)]}}{3}
\]




\end{document}
